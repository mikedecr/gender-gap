% ----------------------------------------------------
%   a preamble file for use with R Markdown
% ----------------------------------------------------

% --- text and typography -----------------------

% not actually sure that microtype works!
% \usepackage{microtype}   % microtypography

% \usepackage{mathptmx} % serif = times (with math)
% \usepackage{helvet} % sens-serif = helvetica clone
% \usepackage[notextcomp]{kpfonts}
\usepackage{libertine}
\usepackage{libertinust1math}
\usepackage[varqu, scaled = 0.95]{zi4} % mono w/ straight quotes

% are these needed is using default latex template??
% \usepackage[utf8]{inputenc} % better interpretation of input characters
% \usepackage[T1]{fontenc}    % better output glyphs/behaviors
% I'm not sure how well these two packages work either!


\usepackage{bm}
\usepackage{setspace}

% --- floats -----------------------
% included in default template:
  % \usepackage{booktabs}
  % \usepackage{longtable}
\usepackage{dcolumn} % decimal-aligned columns
\usepackage{rotating}
\usepackage{placeins}
\usepackage{floatrow}
  \floatsetup[figure]{capposition=top}

% --- Styles -----------------------

% TITLE
\usepackage{titling}

  % title field
  \pretitle{\begin{center} \LARGE}
  \posttitle{\par\end{center}\vskip 12pt}

  % author field
  \preauthor{\begin{center}\large}
  \postauthor{% \\ \normalsize 
              % \emph{University of Wisconsin--Madison}
              \par\end{center}}

  % date field
  \predate{\begin{center}}
  \postdate{\par\end{center}}

% ABSTRACT
% \usepackage{abstract}
%   \renewcommand{\abstractname}{}    % clear the title
%   \renewcommand{\absnamepos}{empty} % originally center

% % SECTIONS
\usepackage[small, bf]{titlesec}
  \titleformat*{\subsection}{\bf \itshape}
  \titleformat*{\subsubsection}{\itshape} 
  \titleformat*{\paragraph}{\itshape} 



% --- Other document logic -----------------------

% to-do notes
\usepackage[colorinlistoftodos, 
            prependcaption, 
            obeyFinal,
            textsize = footnotesize]{todonotes}
  \presetkeys{todonotes}{fancyline, color = violet!30}{}

\usepackage{comment}
% --- bib -----------------------

% natbib declared in RMD
% \usepackage{natbib}
 \bibpunct[: ]{(}{)}{;}{a}{}{,}

% --- commands -----------------------
\newcommand{\notes}[1]{\\
% \raggedright 
\small
\emph{Notes:} #1}
